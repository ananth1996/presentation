\documentclass[pdf]{beamer}
\usetheme{Copenhagen}
\usepackage{multicol, latexsym, amsmath, amssymb}
\usepackage{smartdiagram}
\usepackage{subcaption}

\setbeamertemplate{navigation symbols}{}

\newcommand{\smartart}{   
    \begin{center}
    \scalebox{0.68}{   
        \smartdiagram[constellation diagram:counterclockwise]{
        Updating ML Models,Differential\\Privacy, Optimization, Information\\Theory, Novel\\Pipelines
        }   
    }
    \end{center}
}
  
\newcommand\myNset[1][6]{\ifnum#1=1 \smartdiagramset{set color list={red!60,blue!60,blue!60,blue!60,blue!60},
  uniform connection color=true,
  distance planet-satellite=4.1cm,
  satellite text width=2.5cm,
  planet text width=2.7cm, 
  /tikz/connection planet satellite/.append style={->}
  }
  \else\ifnum#1=2 \smartdiagramset{set color list={blue!60,red!60,blue!60,blue!60,blue!60},
  uniform connection color=true,
  distance planet-satellite=4.1cm,
  satellite text width=2.5cm,
  planet text width=2.75cm, 
  /tikz/connection planet satellite/.append style={->}
  }
  \else\ifnum#1=3 \smartdiagramset{set color list={blue!60,blue!60,red!60,blue!60,blue!60},
  uniform connection color=true,
  distance planet-satellite=4.1cm,
  satellite text width=2.5cm,
  planet text width=2.75cm, 
  /tikz/connection planet satellite/.append style={->}
  }
  \else\ifnum#1=4 \smartdiagramset{set color list={blue!60,blue!60,blue!60,red!60,blue!60},
  uniform connection color=true,
  distance planet-satellite=4.1cm,
  satellite text width=2.5cm,
  planet text width=2.75cm, 
  /tikz/connection planet satellite/.append style={->}
  }
  \else\ifnum#1=5 \smartdiagramset{set color list={blue!60,blue!60,blue!60,blue!60,red!60},
  uniform connection color=true,
  distance planet-satellite=4.1cm,
  satellite text width=2.5cm,
  planet text width=2.75cm, 
  /tikz/connection planet satellite/.append style={->}
  }
  \else\ifnum#1=6 \smartdiagramset{set color list={blue!60,blue!60,blue!60,blue!60,blue!60},
  uniform connection color=true,
  distance planet-satellite=4.1cm,
  satellite text width=2.5cm,
  planet text width=2.75cm, 
  /tikz/connection planet satellite/.append style={->}
  }
  
  \fi\fi\fi\fi\fi\fi}


\newcommand{\dataset}{\mathcal{D}}
\newcommand{\datasetprime}{\mathcal{D}^{\prime}}
\newcommand{\removed}{\mathcal{D_{R}}}
\newcommand{\alg}{A}
\newcommand{\mech}{M}
\newcommand{\hypothesis}{\mathcal{H}}
\newcommand{\subhypothesis}{\mathcal{T}}
\newcommand{\wopt}{\textbf{w}^{*}}
\newcommand{\w}{\textbf{w}}
\newcommand{\wu}{\textbf{w}^{U}}
\newcommand{\wi}{\textbf{w}^{I}}
\newcommand{\h}{\textbf{H}}
\newcommand{\x}{\textbf{x}}
\newcommand{\wminus}{\textbf{w}^{-}}
\newcommand{\risk}{L}
\newcommand{\loss}{\ell}

\DeclareMathOperator*{\argmin}{\arg\!\min}
\DeclareMathOperator*{\argmax}{\arg\!\max}



\title{Updating ML Models}

\author[Ananth Mahadevan]{Ananth Mahadevan}
\date{\today}

\begin{document}
\begin{frame}
    \titlepage
\end{frame}

\begin{frame}
    \frametitle{Overview}
    \tableofcontents
\end{frame}

\section{Motivation}

\section{Problem Overview}

\section{Approaches}
\begin{frame}
  \frametitle{}
  \myNset[6]
  \smartart
\end{frame}

\begin{frame}
  \frametitle{Common Terminology}
  \begin{itemize}
    \item Fixed training Dataset $\dataset$
    \item Learning Algorithm $\alg$ (can be randomized)
    \item Datapoints to be remove $\removed$, where $|\removed|=r$, remaining dataset $\datasetprime=\dataset - \removed$
    \item Naive approach is retraining from scratch, i.e, $\alg(\datasetprime)$
    \item Mechanism $\mech$ which offers an efficient way to update the model
  \end{itemize}
\end{frame}

\subsection{Differential Privacy}
\begin{frame}
  \myNset[1]
  \smartart
\end{frame}

\begin{frame}
  \frametitle{Certified Data Removal \cite{guoCertifiedDataRemoval2020}}
  \begin{itemize}
    \item $\alg$ outputs a model in hypothesis space $\hypothesis$
    \item Defines $\epsilon$-\textit{certified removal}, $\forall \subhypothesis \subseteq \hypothesis$
    \[
      e^{-\epsilon} \le \frac{P(M(\alg(\dataset),\removed)\in \subhypothesis)}{P(A(\datasetprime)\in \subhypothesis)} \le e^{\epsilon}
    \]
    \item Insufficiency of Parametric indistinguishability
    \begin{itemize}
      \item Approximate removal processes leaves a gradient residual 
      \item Residuals can reveal the prior presence of that training sample
    \end{itemize}
  \end{itemize}
\end{frame}

\begin{frame}
  \frametitle{Removal Mechanism for Linear Classfiers}
  \begin{itemize}
    \item $\alg$ empirical risk $\risk(\w;\dataset)$ with a convex loss function $\loss(\w^{T}\x,y)$
    \item $\wopt = \alg(\dataset)=\argmin_{w}\risk(\w;\dataset)$
    \item To remove a single point $\removed = \{(\x_{n},y_{n})\}$ 
    \item Newton Update Step: $\wminus=\mech(\wopt,(\x_{n},y_{n})) = \wopt - H^{-1}_{\wopt}\nabla$
    \item Where $H_{\wopt} = \nabla^{2}\risk(\wopt,\datasetprime)$ and $\nabla = \lambda\wopt + \nabla\loss((\wopt)^{T}\x_{n},y_{n})$
    \item $H ^{-1}_{\wopt}\nabla$ is from \textit{influence function} literature 
  \end{itemize}

\end{frame}

\begin{frame}
  \frametitle{Influence Function}
  \includegraphics[width=\textwidth]{images/influence functions.pdf}
\end{frame}

\begin{frame}
  \frametitle{Certifing Removal}
  \begin{itemize}
    \item $\wminus$ is approximate close to minimizer of $\risk(\w;\datasetprime)$
    \item $\nabla\risk(\wminus;\datasetprime)$ is gradient residual and if non-zero, reveals Information
    \item Even a small $\|\nabla\risk(\wminus;\datasetprime)\|_{2}$ doesn't guarantee certifiable removal 
    \item Therefore, perturb loss at training time 
    \[
      \risk_{b}(\w;\dataset) = \sum_{i=1}^{n}\loss(\w^{T}\x_{i},y_{i})+\frac{\lambda n}{2}\|\w\|_{2}^{2} + \textbf{b}^{T}\w
    \]
    Where $\textbf{b}\in \mathbb{R}^{d}$ drawn randomly from some distribution
  \end{itemize}

\end{frame}

\begin{frame}
  \frametitle{Benefits and Drawbacks}
  \begin{block}{Benefits}
    \begin{itemize}
      \item Provides formal guarantee of statistical indistinguishability
      \item Works well with Differentially Private trained networks
      \item Uses influence functions to approximate data removal
    \end{itemize}    
  \end{block}
  \begin{alertblock}{Limitations}
    \begin{itemize}
      \item Requires inverting a Hessian matrix
      \item Non-convex loss functions not supported 
      \item Adding noise during training hurts model performance
      \item Very strict notion of removal
    \end{itemize}
    
  \end{alertblock}
\end{frame}
\subsection{Optimization}
\begin{frame}
  \myNset[2]
  \smartart
\end{frame}

\begin{frame}
  \frametitle{
    DeltaGrad \cite{wuDeltaGradRapidRetraining2020}
    }
  \begin{itemize}
    \item $\mech$ targets the Gradient Descent (GD) algorithm
    \item Naive retraining $\alg(\datasetprime)$ recomputed gradients over all remaining points
    \[
        \wu_{t+1} \leftarrow \wu_{t} - \frac{\eta_{t}}{n-r}\sum_{i\in \datasetprime}\nabla \risk_{i}(\wu_{t})
    \]
    \item Instead rewrite it as a \textit{leave-r-out} formula
    \[
      \wi_{t+1} = \wi_{t} - \frac{\eta_{t}}{n-r}\left[\sum_{i\in \dataset}\nabla \risk_{i}(\wi_{t}) -\sum_{i\in \removed} \nabla \risk_{i}(\wi_{t})\right]
    \]
    \item Much cheaper to compute $r$ gradients, when $r \ll n$
  \end{itemize}

\end{frame}

\begin{frame}
  \frametitle{Approximating $\sum_{i\in \dataset}\nabla \risk_{i}(\wi_{t})$}
  \begin{itemize}
    \item Need to use historical $\nabla \risk(\w_{t})$ to approximate $\nabla\risk(\wi_{t})$
    \item Taylor expansion around $\wi_{t}$ gives the following
    \[
       \nabla \risk(\wi_{t}) = \nabla \risk(\w_{t}) + \h_{t}\cdot(\wi_{t}-\w_{t}) 
    \]
    Where $\h_{t} = \int_{0}^{1}\h(\w_{t}+x(\wi_{t}-\w))dx$
    \item Maintaining a Hessian matrix is expensive, so leverage the L-BFGS algorithm to compute a Hessian-vector product
    \item This leads to issues in error bounds of the approximation
  \end{itemize}
    
\end{frame}

\begin{frame}
  \frametitle{Problem with Error Bound}
  \includegraphics[page=43,clip,trim=0.5cm 1cm 0cm 1cm,width=\textwidth]{images/Slides.pdf}
\end{frame}

\begin{frame}
  \frametitle{Controlling the Errors}
  \begin{itemize}
    \item Do explicit evaluations for $j_{0}$ "burn-in" iterations and then periodically every $T_{0}$ iterations
    \includegraphics[page=46,clip,trim=0.5cm 1cm 0cm 1cm,width=0.8\textwidth]{images/Slides.pdf}
  \end{itemize}
\end{frame}

\begin{frame}
  \frametitle{Benefits and Limitations}
  \begin{block}{Benefits}
    \begin{itemize}
      \item Handles both additions nad deletions of datapoints
      \item Can be applied to any ML model trained using Stochastic Gradient Descent
      \item Approximation guarantees and empirical results on 
    \end{itemize}
  \end{block}
  \begin{alertblock}{Limitaitons}
    \begin{itemize}
      \item Needs to cache all weights $\w_{t}$ and gradients $\nabla \risk(\w_{t})$ during training
      \item Requires tuning of $T_{0}$ and $j_{0}$ based on dataset
      \item For SGD, only works with large batch sizes ($>10000$), which hurts model performance 
    \end{itemize}
  \end{alertblock}
  

\end{frame}

\subsection{Information Theory}
\begin{frame}
  \myNset[3]
  \smartart
\end{frame}

\begin{frame}
  \frametitle{
    Eternal Sunshine of the Spotless Net \cite{golatkarEternalSunshineSpotless2020}
    }
  \begin{itemize}
    \item $P(\w | \dataset)$ distribution of algorithm $\alg$
    \item $\mech$ is called \emph{scrubbing function} applied to $\w$
    \item $P(M(\w)|\dataset)$ is distribution of possible weights after scrubbing
    \item Motivation: disallow attacker to use \textit{read-out function} $f(\w)$ to gain information about $\removed$
    \item Therefore, optimal scrubbing function must have 
    \[
      \KL(P(f(\mech(\w))|\dataset)~\|~P(f(S_{0}(\w))|\datasetprime))=0
    \]
    Where $S_{0}$ is a \textit{certificate} of forgetting
    \item To be agnostic of $f(\cdot)$, minimize
    \[
      \KL(P(\mech(\w)|\dataset)~\|~P(S_{0}(\w)|\datasetprime)) 
    \]
  \end{itemize}
\end{frame}

\begin{frame}
  \frametitle{Forgetting Lagrangian}
  \begin{itemize}
    \item A trivial noise scrubbing is $M(\w)=S_{0}(\w) = w+\sigma n$, where $n \sim \mathcal{N}(0,I)$
    \item As $\sigma \rightarrow \infty$, $\KL(p\|q)\rightarrow 0$, which invalidates the model 
    \item Define the \emph{Forgetting Lagrangian}:
    \[
      \mathcal{L} = \mathbb{E}_{\mech(\w)}[\risk_{\datasetprime}(\w)] + \lambda\KL(P(\mech(\w)|\dataset)~\|~P(S_{0}(\w)|\datasetprime)) 
    \]
    \item Use quadratic approximation and noise to scrub weights
    \item $\mech(\w) = h(\w)+n$ and $S_{0}=w + n^{\prime}$ where $h(\w)$ is deterministic and $n,n^{\prime} \sim \mathcal{N}(,\Sigma)$
  \end{itemize}

\end{frame}

\begin{frame}
  \frametitle{Scrubbing Example }
  \begin{center}
  \includegraphics[width=0.9\textwidth]{images/scrubbing.pdf}
  \end{center}

\end{frame}
\begin{frame}
  \frametitle{Robust Quadratic Scrubbing}
  \begin{itemize}
    \item Noisy Newton update, as $t\rightarrow\infty$ is defined as 
    \[
      \mech_{t}(\w) = \w - \h^{-1}\nabla\risk_{\datasetprime}(\w) + (\lambda\sigma^{2}_{h})^{1/4}\h^{-1/4}
    \]
    Where $\h = \nabla^{2}(\risk_{\datasetprime}(\w))$, $\sigma_{h}$ represents error in approximating SGD with a continuous gradient flow and $\lambda$ hyperparameter
    \item For Deep Neural Networks, Hessian matrix is expensive to compute and store
    \item Simplified scrubbing to only adding noise 
    \[
      \mech(\w) = \w + (\lambda\sigma^{2}_{h})^{1/4}F^{-1/4}
    \]
    \item $F$ is the Fisher Information Matrix, computed using the Levenberg- Marquardt semi-positive-definite approximation of $\nabla^{2}\risk_{\dataset}(\w)$
  \end{itemize}
  
\end{frame}
\begin{frame}
  \frametitle{Benefits and Limitations}
  \begin{block}{Benefits}
    \begin{itemize}
      \item Works for Deep Neural Networks
      \item Allows to remove a entire class, multiple classes, or a subset of a class of the training dataset
      \item Process is optimal if quadratic assumptions hold true
    \end{itemize}
  \end{block}

  \begin{alertblock}{Limitations}
    \begin{itemize}
      \item Space and time complexity of approach unknown
      \item Considers worst case of attacker using any read-out function $f(\cdot)$
      \item Results based on stability of SGD after pre-training networks
    \end{itemize}
  \end{alertblock}
\end{frame}

\subsection{Novel Pipelines}
\begin{frame}
  \myNset[4]
  \smartart
\end{frame}


\begin{frame}
  \frametitle{
    Machine Unlearning: SISA \cite{bourtouleMachineUnlearning2020}
    }
  \begin{itemize}
    \item 
  \end{itemize}
\end{frame}

\section{Next Directions}



\begin{frame}[allowframebreaks]
  \frametitle{References}
  \bibliography{UpdateMl}
  \bibliographystyle{apalike}
\end{frame}



\end{document}  