\documentclass[pdf]{beamer}
\usetheme{Copenhagen}
\usepackage{multicol, latexsym, amsmath, amssymb}
\usepackage{smartdiagram}
\usepackage{subcaption}

\setbeamertemplate{navigation symbols}{}

\newcommand{\smartart}{   
    \begin{center}
    \scalebox{0.68}{   
        \smartdiagram[constellation diagram:counterclockwise]{
        Updating ML Models,Differential\\Privacy, Optimization, Information\\Theory, Novel\\Pipelines
        }   
    }
    \end{center}
}
  
\newcommand\myNset[1][6]{\ifnum#1=1 \smartdiagramset{set color list={red!60,blue!60,blue!60,blue!60,blue!60},
  uniform connection color=true,
  distance planet-satellite=4.1cm,
  satellite text width=2.5cm,
  planet text width=2.7cm, 
  /tikz/connection planet satellite/.append style={->}
  }
  \else\ifnum#1=2 \smartdiagramset{set color list={blue!60,red!60,blue!60,blue!60,blue!60},
  uniform connection color=true,
  distance planet-satellite=4.1cm,
  satellite text width=2.5cm,
  planet text width=2.75cm, 
  /tikz/connection planet satellite/.append style={->}
  }
  \else\ifnum#1=3 \smartdiagramset{set color list={blue!60,blue!60,red!60,blue!60,blue!60},
  uniform connection color=true,
  distance planet-satellite=4.1cm,
  satellite text width=2.5cm,
  planet text width=2.75cm, 
  /tikz/connection planet satellite/.append style={->}
  }
  \else\ifnum#1=4 \smartdiagramset{set color list={blue!60,blue!60,blue!60,red!60,blue!60},
  uniform connection color=true,
  distance planet-satellite=4.1cm,
  satellite text width=2.5cm,
  planet text width=2.75cm, 
  /tikz/connection planet satellite/.append style={->}
  }
  \else\ifnum#1=5 \smartdiagramset{set color list={blue!60,blue!60,blue!60,blue!60,red!60},
  uniform connection color=true,
  distance planet-satellite=4.1cm,
  satellite text width=2.5cm,
  planet text width=2.75cm, 
  /tikz/connection planet satellite/.append style={->}
  }
  \else\ifnum#1=6 \smartdiagramset{set color list={blue!60,blue!60,blue!60,blue!60,blue!60},
  uniform connection color=true,
  distance planet-satellite=4.1cm,
  satellite text width=2.5cm,
  planet text width=2.75cm, 
  /tikz/connection planet satellite/.append style={->}
  }
  
  \fi\fi\fi\fi\fi\fi}


\newcommand{\dataset}{\mathcal{D}}
\newcommand{\datasetprime}{\mathcal{D}^{\prime}}
\newcommand{\removed}{\mathcal{D_{R}}}
\newcommand{\alg}{A}
\newcommand{\mech}{M}
\newcommand{\hypothesis}{\mathcal{H}}
\newcommand{\subhypothesis}{\mathcal{T}}
\newcommand{\wopt}{\textbf{w}^{*}}
\newcommand{\w}{\textbf{w}}
\newcommand{\wu}{\textbf{w}^{U}}
\newcommand{\wi}{\textbf{w}^{I}}
\newcommand{\h}{\textbf{H}}
\newcommand{\x}{\textbf{x}}
\newcommand{\wminus}{\textbf{w}^{-}}
\newcommand{\risk}{L}
\newcommand{\loss}{\ell}

\DeclareMathOperator*{\argmin}{\arg\!\min}
\DeclareMathOperator*{\argmax}{\arg\!\max}



\title{Updating ML Models}

\author[Ananth Mahadevan]{Ananth Mahadevan}
\date{\today}

\begin{document}
\begin{frame}
    \titlepage
\end{frame}

\begin{frame}
    \frametitle{Overview}
    \tableofcontents
\end{frame}

\section{Motivation}

\section{Problem Overview}

\section{Approaches}
\begin{frame}
  \frametitle{}
  \myNset[6]
  \smartart
\end{frame}

\begin{frame}
  \frametitle{Common Terminology}
  \begin{itemize}
    \item Fixed training Dataset $\dataset$
    \item Learning Algorithm $\alg$ (can be randomized)
    \item Datapoints to be remove $\removed$, where $|\removed|=r$, remaining dataset $\datasetprime=\dataset - \removed$
    \item Naive approach is retraining from scratch, i.e, $\alg(\datasetprime)$
    \item Mechanism $\mech$ which offers an efficient way to update the model
  \end{itemize}
\end{frame}

\subsection{Differential Privacy}
\begin{frame}
  \myNset[1]
  \smartart
\end{frame}

\begin{frame}
  \frametitle{Certified Data Removal \cite{guoCertifiedDataRemoval2020}}
  \begin{itemize}
    \item $\alg$ outputs a model in hypothesis space $\hypothesis$
    \item Defines $\epsilon$-\textit{certified removal}, $\forall \subhypothesis \subseteq \hypothesis$
    \[
      e^{-\epsilon} \le \frac{P(M(\alg(\dataset),\removed)\in \subhypothesis)}{P(A(\datasetprime)\in \subhypothesis)} \le e^{\epsilon}
    \]
    \item Insufficiency of Parametric indistinguishability
    \begin{itemize}
      \item Approximate removal processes leaves a gradient residual 
      \item Residuals can reveal the prior presence of that training sample
    \end{itemize}
  \end{itemize}
\end{frame}

\begin{frame}
  \frametitle{Removal Mechanism for Linear Classfiers}
  \begin{itemize}
    \item $\alg$ empirical risk $\risk(\w;\dataset)$ with a convex loss function $\loss(\w^{T}\x,y)$
    \item $\wopt = \alg(\dataset)=\argmin_{w}\risk(\w;\dataset)$
    \item To remove a single point $\removed = \{(\x_{n},y_{n})\}$ 
    \item Newton Update Step: $\wminus=\mech(\wopt,(\x_{n},y_{n})) = \wopt - H^{-1}_{\wopt}\nabla$
    \item Where $H_{\wopt} = \nabla^{2}\risk(\wopt,\datasetprime)$ and $\nabla = \lambda\wopt + \nabla\loss((\wopt)^{T}\x_{n},y_{n})$
    \item $H ^{-1}_{\wopt}\nabla$ is from \textit{influence function} literature 
  \end{itemize}

\end{frame}

\begin{frame}
  \frametitle{Influence Function}
  \includegraphics[width=\textwidth]{images/influence functions.pdf}
\end{frame}

\begin{frame}
  \frametitle{Certifing Removal}
  \begin{itemize}
    \item $\wminus$ is approximate close to minimizer of $\risk(\w;\datasetprime)$
    \item $\nabla\risk(\wminus;\datasetprime)$ is gradient residual and if non-zero, reveals Information
    \item Even a small $\|\nabla\risk(\wminus;\datasetprime)\|_{2}$ doesn't guarantee certifiable removal 
    \item Therefore, perturb loss at training time 
    \[
      \risk_{b}(\w;\dataset) = \sum_{i=1}^{n}\loss(\w^{T}\x_{i},y_{i})+\frac{\lambda n}{2}\|\w\|_{2}^{2} + \textbf{b}^{T}\w
    \]
    Where $\textbf{b}\in \mathbb{R}^{d}$ drawn randomly from some distribution
  \end{itemize}

\end{frame}

\subsection{Optimization}
\begin{frame}
  \myNset[2]
  \smartart
\end{frame}

\begin{frame}
  \frametitle{
    DeltaGrad \cite{wuDeltaGradRapidRetraining2020}
    }
  \begin{itemize}
    \item 
  \end{itemize}
\end{frame}


\subsection{Database Based}
\begin{frame}
  \myNset[3]
  \smartart
\end{frame}


\begin{frame}
  \frametitle{
    PrIU \cite{wuPrIUProvenanceBasedApproach2020}
    }
  \begin{itemize}
    \item 
  \end{itemize}
\end{frame}

\subsection{Information Theory}
\begin{frame}
  \myNset[4]
  \smartart
\end{frame}

\begin{frame}
  \frametitle{
    Eternal Sunshine of the Spotless Net \cite{golatkarEternalSunshineSpotless2020}
    }
  \begin{itemize}
    \item 
  \end{itemize}
\end{frame}

\subsection{Novel Pipelines}
\begin{frame}
  \myNset[5]
  \smartart
\end{frame}


\begin{frame}
  \frametitle{
    Machine Unlearning: SISA \cite{bourtouleMachineUnlearning2020}
    }
  \begin{itemize}
    \item 
  \end{itemize}
\end{frame}

\section{Next Directions}



\begin{frame}[allowframebreaks]
  \frametitle{References}
  \bibliography{UpdateMl}
  \bibliographystyle{apalike}
\end{frame}



\end{document}  