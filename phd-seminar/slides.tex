\documentclass[pdf]{beamer}
\usetheme{Copenhagen}
\usepackage{pgfgantt}
% Need to use LuaLatex to compile
\usepackage{emoji}
\usepackage{booktabs}
\usepackage{tikz}
\usepackage{fontawesome}
\usepackage{amsmath}
\usepackage{natbib}
\usepackage[most]{tcolorbox}
\usetikzlibrary{angles,calc,intersections,quotes,arrows.meta,positioning}
\setemojifont{Apple Color Emoji}
\setbeamercovered{transparent}
\beamertemplatenavigationsymbolsempty
\setbeamertemplate{footline}[text line]{%
\parbox{\linewidth}{\vspace*{-8pt}\hfill\hfill\insertframenumber\,/\,\inserttotalframenumber}}

\title{``$2^{\text{nd}}$'' Year PhD Seminar Presentation}

\author[Ananth Mahadevan]{Ananth Mahadevan}
\date{\today}
\usepackage{xspace}
\newcommand*{\error}{\ensuremath{\accDis}\xspace}
\newcommand*{\accuracy}{\ensuremath{\acctest}\xspace}

% training 
\newcommand*{\D}{\ensuremath{\mathcal{D}}\xspace}
\newcommand*{\sample}{\ensuremath{\mathbf{x}}\xspace}
\newcommand*{\dimensions}{\ensuremath{d}\xspace}
\newcommand*{\labels}{\ensuremath{\mathbf{y}}\xspace}

\newcommand*{\Dtrain}{\ensuremath{\D_{\text{init} }}\xspace}
\newcommand*{\Dm}{\ensuremath{\mathcal{D}_{\nremovals}}\xspace}
\newcommand*{\Dprime}{\ensuremath{\D\setminus\Dm}\xspace}
\newcommand*{\Dtest}{\ensuremath{\mathcal{D}_{\text{test}}}\xspace}
\newcommand*{\nremovals}{\ensuremath{m}\xspace}
\newcommand*{\ntrain}{\ensuremath{n_{\text{init}}}\xspace}
\newcommand*{\ntest}{\ensuremath{n}_{\text{test}}\xspace}
\newcommand*{\obj}{\ensuremath{L}\xspace}

%
\newcommand*{\UM}{\ensuremath{u}\xspace}
\newcommand*{\ml}{ML\xspace}
\newcommand*{\sgd}{SGD\xspace}
\newcommand*{\w}{\ensuremath{\mathbf{w}}\xspace}
\newcommand*{\wopt}{\ensuremath{\w^{opt}}\xspace}
\newcommand*{\worig}{\ensuremath{\mathbf{w}^{*}}\xspace}
\newcommand*{\employed}{\ensuremath{e}\xspace}
\newcommand*{\wEmployed}{\ensuremath{\w}\xspace}
\newcommand*{\wunlearned}{\ensuremath{\w^\UM}\xspace}
\newcommand*{\wprime}{\ensuremath{\mathbf{w}^{\prime}}\xspace}
% 

\newcommand*{\efficiencyParameter}{\ensuremath{\tau}\xspace}
\newcommand*{\QoA}{\ensuremath{\efficiencyParameter}\xspace}
\newcommand*{\noiseParamter}{\ensuremath{\sigma}\xspace}
\newcommand*{\unlrbatchsize}{\ensuremath{\nremovals^{\prime}}\xspace}
% Guo et al. 
\newcommand*{\noisyobj}{\ensuremath{\obj_{\sigma}}\xspace}
\newcommand*{\infl}{\textsc{Influence}\xspace}
\newcommand*{\guo}{\textsc{guo}\xspace}

% Golatkar et al.
\newcommand*{\fisher}{\textsc{Fisher}\xspace}
\newcommand*{\fMatrix}{\ensuremath{F}\xspace}
\newcommand*{\gol}{\textsc{gol}\xspace}

% Wu et al. 
\newcommand*{\wu}{\textsc{wu}\xspace}
\newcommand*{\deltagrad}{\textsc{DeltaGrad}\xspace}
\newcommand*{\iw}{\ensuremath{\w}\xspace}
\newcommand*{\dgapprox}{\textsc{DGApprox}\xspace}

% Distribution Experiment
\newcommand*{\classprob}{p_{\text{class}}\xspace}
\newcommand*{\infoprob}{p_{\text{info}}\xspace}

% Datasets

\newcommand*{\bmnist}{\ensuremath{\text{\sc mnist}^{\text{b}}}\xspace}
\newcommand*{\mnist}{\text{\sc mnist}\xspace}
\newcommand*{\covtype}{\text{\sc covtype}\xspace}
\newcommand*{\higgs}{\text{\sc higgs}\xspace}
\newcommand*{\cifar}{\text{\sc cifar2}\xspace}
\newcommand*{\eps}{\text{\sc epsilon}\xspace}
\newcommand*{\T}{\top}
\newcommand*{\x}{\mathbf{x}\space}
\newcommand*{\noiseb}{\ensuremath{\mathbf{b}}\xspace}

% metrics
\newcommand{\smape}{\ensuremath{\text{\tt SMAPE}}\xspace}
\newcommand{\sape}{\ensuremath{\text{\tt SAPE}}\xspace}
\newcommand{\acctest}{\ensuremath{\text{\tt Acc}_{\text{test}}}\xspace}
\newcommand{\accremoved}{\ensuremath{\text{\tt Acc}_{\text{del}}}\xspace}
\newcommand*{\accErr}{\texttt{AccErr}\xspace}
\newcommand*{\accDrop}{\accErr}
\newcommand*{\relEff}{\accDrop}
\newcommand*{\accDis}{\texttt{AccDis}\xspace}

% sampling types 

\newcommand{\unirand}{\text{{\tt uniform-random}}\xspace}
\newcommand{\targrand}{\text{{\tt targeted-random}}\xspace}
\newcommand{\uniinfo}{\text{{\tt uniform-informed}}\xspace}
\newcommand{\targinfo}{\text{{\tt targeted-informed}}\xspace}

% When to retrain
\newcommand*{\winit}{\ensuremath{\worig_{\text{init}}}\xspace}
\newcommand*{\accdropinit}{\ensuremath{\accDrop_{\text{init}}}\xspace}
\newcommand*{\acctestinit}{\ensuremath{\acctest^{\text{init}}}\xspace}
\newcommand*{\slope}{\ensuremath{c}\xspace}
\newcommand*{\disPred}{\ensuremath{\overline{\error}}\xspace}
\newcommand*{\estErr}{\ensuremath{EstErr}\xspace}
\newcommand*{\estDis}{\ensuremath{EstDis}\xspace}
\newcommand*{\thresh}{\ensuremath{\kappa}\xspace}

%
\newcommand*{\repo}{\url{https://version.helsinki.fi/mahadeva/unlearning-experiments}}

\newcommand*{\good}{\ensuremath{\uparrow}}
\newcommand*{\better}{\ensuremath{\uparrow\uparrow}}
\newcommand*{\best}{\ensuremath{\uparrow\uparrow\uparrow}}
\newcommand{\revision}[1]{\textcolor{blue}{{#1}}}

\newcommand*{\inc}{\ensuremath{\textcolor{green!80!black}\uparrow}}
\newcommand*{\dec}{\ensuremath{\textcolor{red!80!black}\downarrow}}







\usepackage{xspace}
\usepackage{amsmath}
\usepackage{amssymb}
\usepackage{subcaption}



\newcommand{\norm}[1]{\|#1\|}
\newcommand{\inner}[2]{\langle #1, #2 \rangle}
\newcommand{\round}[1]{\left(#1\right)}
\newcommand{\braces}[1]{\left\{#1\right\}}
\newcommand{\squares}[1]{\left[#1\right]}
\newcommand{\product}[1]{\left\langle#1\right\rangle}

% \newcommand*{\ml}{ML}

\newcommand*{\decision}{\ensuremath{\mathcal{R}}\xspace}
% \newcommand*{\QDM}{\ensuremath{\text{\sc QDM}}\xspace}
\newcommand*{\QDMq}{\ensuremath{\psi}\xspace}
\newcommand*{\QDM}{\ensuremath{\Psi}\xspace}
% \newcommand*{\QDMdiff}{\ensuremath{\QDM_{\text{diff}}}\xspace}
\newcommand*{\QDMdiff}{\ensuremath{\overline{\Psi}}\xspace}
\newcommand*{\simfn}{\ensuremath{\text{sim}}\xspace}
\newcommand*{\loss}{\ensuremath{\ell}\xspace}
\newcommand*{\tprime}{\ensuremath{t^{\prime}}\xspace}
\newcommand*{\feat}{\ensuremath{d}\xspace}
\newcommand*{\q}{\ensuremath{q}\xspace}
% \newcommand*{\x}{\ensuremath{x}\xspace}
\newcommand*{\y}{\ensuremath{y}\xspace}
\newcommand*{\X}{\ensuremath{\mathbf{X}}\xspace}
\newcommand*{\Y}{\ensuremath{\mathbf{y}}\xspace}
\newcommand*{\Yset}{\ensuremath{\mathcal{Y}}\xspace}
\newcommand*{\Xset}{\ensuremath{\mathcal{X}}\xspace}
\newcommand*{\ttime}{\ensuremath{t}\xspace}
\newcommand*{\data}{\ensuremath{D}\xspace}
\newcommand*{\model}{\ensuremath{M}\xspace}
\newcommand*{\query}{\ensuremath{Q}\xspace}
\newcommand*{\Dtprime}{\ensuremath{\data_{\tprime}}\xspace}
\newcommand*{\Mtprime}{\ensuremath{\model_{\tprime}}\xspace}
\newcommand*{\Qtprime}{\ensuremath{\query_{\tprime}}\xspace}
\newcommand*{\Dt}{\ensuremath{\data_{t}}\xspace}
\newcommand*{\Mt}{\ensuremath{\model_{t}}\xspace}
\newcommand*{\Qt}{\ensuremath{\query_{t}}\xspace}
\newcommand*{\dataBatch}{\ensuremath{B_\data}\xspace}
\newcommand*{\queryBatch}{\ensuremath{B_\query}\xspace}
% \newcommand*{\T}{\ensuremath{T}\xspace}
\newcommand*{\offline}{\text{offline}\xspace}
\newcommand*{\online}{\text{online}\xspace}
\newcommand*{\Toffline}{\ensuremath{\T_{\offline}}\xspace}
\newcommand*{\Tonline}{\ensuremath{\T_{\online}}\xspace}
\newcommand*{\Tstart}{\ensuremath{\T_{\text{start}}}\xspace}
\newcommand*{\Tend}{\ensuremath{\T_{\text{end}}}\xspace}
\newcommand*{\Cost}{\ensuremath{C}\xspace}
\newcommand*{\totalcost}{\ensuremath{c}\xspace}
\newcommand*{\cost}[1]{\ensuremath{\text{cost}\round{#1}}\xspace}
\newcommand*{\dpcost}[2][]{%
    \ifx\\#1\\%
        \ensuremath{v\round{#2}}\xspace%
        \else%
        \ensuremath{v^{#1}\round{#2}}\xspace%
    \fi%
}
\newcommand*{\prev}{\ensuremath{p}\xspace}
\newcommand*{\dptable}{\ensuremath{V}\xspace}
\newcommand*{\costmatrix}{\Cost}
\newcommand*{\costentry}[1]{\ensuremath{c_{#1}}\xspace}
\newcommand*{\strategy}{\ensuremath{S}\xspace}
\newcommand*{\partialstrategy}{\ensuremath{\bar{\strategy}}\xspace}
\newcommand*{\partialstrategyopt}{\ensuremath{\round{\partialstrategy}^{*}}\xspace}

\newcommand*{\oracleretrains}{\ensuremath{O}\xspace}
\newcommand*{\retraincost}{\ensuremath{\kappa}\xspace}

\newcommand*{\retrain}{\texttt{Retrain}\xspace}
\newcommand*{\keep}{\texttt{Keep}\xspace}


%algorithms

\newcommand*{\oracle}{\textsc{Oracle}\xspace}
\newcommand*{\neverretrain}{\textsc{NR}\xspace}
\newcommand*{\algo}{\textsc{Cara}\xspace}
\newcommand*{\cara}{\algo}
\newcommand*{\algoparams}{\ensuremath{\theta}\xspace}
\newcommand*{\algoparamsopt}{\ensuremath{\algoparams^{*}}\xspace}

\newcommand*{\markov}{\textsc{Markov}\xspace}
\newcommand*{\algoMarkov}{\textsc{\algo-M}\xspace}
\newcommand*{\algoThresh}{\textsc{\algo-T}\xspace}
% \newcommand*{\thresh}{\ensuremath{\tau}\xspace}
\newcommand*{\threshopt}{\ensuremath{\thresh^{*}}\xspace}
\newcommand*{\cumthresh}{\ensuremath{\tau_\text{cum}}\xspace}
\newcommand*{\cumthreshopt}{\ensuremath{\cumthresh^{*}}\xspace}

\newcommand*{\algoCummThresh}{\textsc{\algo-CT}\xspace}
\newcommand*{\algoPeriod}{\textsc{\algo-P}\xspace}
\newcommand*{\period}{\ensuremath{\phi}\xspace}
\newcommand*{\periodopt}{\ensuremath{\period^{*}}\xspace}
\newcommand*{\offset}{\ensuremath{a}\xspace}
\newcommand*{\offsetopt}{\ensuremath{\offset^{*}}\xspace}


%baselines
\newcommand*{\adwin}{\textsc{ADWIN}\xspace}
\newcommand*{\ddm}{\textsc{DDM}\xspace}

%metrics 
\newcommand*{\scpe}{\texttt{SCPE}\xspace}
%datasets
\newcommand*{\covcon}{\textsc{CovCon}\xspace}
\newcommand*{\covconData}{\textsc{\covcon-D}\xspace}
\newcommand*{\covconStatic}{\textsc{\covcon-S}\xspace}
\newcommand*{\Circle}{\textsc{Circle}\xspace}
\newcommand*{\Circles}{\Circle}
\newcommand*{\circles}{\Circle}
\newcommand*{\circlesData}{\textsc{\Circle-D}\xspace}
\newcommand*{\circlesStatic}{\textsc{\Circle-S}\xspace}
\newcommand*{\gauss}{\textsc{Gauss}\xspace}
\newcommand*{\gaussData}{\textsc{\gauss-D}\xspace}
\newcommand*{\gaussStatic}{\textsc{\gauss-S}\xspace}
\newcommand*{\elec}{\textsc{Electricity}\xspace}
\newcommand*{\covertype}{\textsc{Covertype}\xspace}
% \newcommand*{\covtype}{\textsc{Covtype}\xspace}
\newcommand*{\airlines}{\textsc{Airlines}\xspace}
\newcommand*{\fixedretraincost}{\ensuremath{46}}


\begin{document}
\begin{frame}
    \titlepage
\end{frame}

\begin{frame}
    \frametitle{Quick Recap}
    \begin{itemize}
        \item \textbf{Masters:} Aalto University
        \item \textbf{Started:} Aug 2020 (contract) and Jan 2021 (study right) 
        \item \textbf{Supervisor:} Michael Mathioudakis
        \item \textbf{Research Group:} Algorithmic Data Science (ADS)
    \end{itemize}

    

\end{frame}

\begin{frame}[fragile]
    \frametitle{PhD Progress}
    \begin{ganttchart}[
        % hgrid,
        % vgrid,
        bar/.append style={fill=gray!50},
        time slot format=isodate-yearmonth,
        time slot unit=month,
        x unit=1.5mm,
        y unit chart=7mm,
        % y unit=1mm,
        progress label text={\pgfmathprintnumber[precision=0, verbatim]{#1}\%},
        % progress=today,
        today=2024-03,
        bar height=.5,
        group peaks width=1,
        group peaks height=0.4,
        rejected/.style={milestone/.append style={fill=red}},
        published/.style={milestone/.append style={fill=green}},
        review/.style={milestone/.append style={fill=yellow}},
        futurepublished/.style={milestone/.append style={fill=gray!50}},
        % bar top shift=.5
      ]{2020-01}{2024-12}
    \gantttitlecalendar{year} \\
      \ganttgroup[progress=today]{PhD}{2020-08}{2024-12}\\
      \ganttbar{Paper I}{2021-01}{2021-06} % Unlearning 
      \ganttmilestone[rejected]{}{2021-07} % VLDB-2022
      \ganttmilestone[rejected]{}{2022-02} % ECML_PKDD
      \ganttmilestone[published]{}{2022-04}\\ % MDPI-MAKE
      \ganttbar{Paper II}{2022-01}{2022-02} % JANE
      \ganttmilestone[rejected]{}{2022-03} % ANS extension 2022
      \ganttmilestone[published]{}{2022-06}\\
      \ganttbar{Paper III}{2020-08}{2020-09} % Sketching
      \ganttbar{}{2021-06}{2021-09} % Sketching
      \ganttbar{}{2022-03}{2022-05} % Sketching
      \ganttmilestone[published]{}{2022-09}\\ % CIKM 2022
      \ganttbar{Paper IV}{2022-05}{2023-02} % Reception Reader
      \ganttmilestone[published]{}{2023-04}\\ % JOHD 2023
      \ganttbar{Paper V}{2022-11}{2023-04} % Mandeville
      \ganttmilestone[published]{}{2023-12}\\ % JOHD 2023
      \ganttbar{Paper VI}{2021-11}{2022-12} % UpdateML
      \ganttmilestone[rejected]{}{2023-03} % ICML 2023
      \ganttmilestone[rejected]{}{2023-08} % EuroSys 2024
      \ganttmilestone[review]{}{2024-02}\\ % KBS 2024
      \ganttbar{Paper VII}{2023-06}{2024-02} % TextReuse Pipeline
      \ganttmilestone[futurepublished]{}{2024-04}\\ % KBS 2024

    \end{ganttchart}    

\end{frame}


\begin{frame}
    \frametitle{Papers}
    \begin{table}
        \centering
        \begin{tabular}{cccc}
            \toprule
            Number & One Word Title & Venue & Include in Thesis? \\
            \midrule
            I & Unlearning & MAKE 2022  & \emoji{check-mark-button}  \\
            II & JANE & Entropy 2022   &\emoji{check-mark-button} \\
            III & Sketching& CIKM 2022 & \emoji{man-shrugging-medium-skin-tone}\\
            IV & ReceptionReader& JOHD 2023 &  \emoji{man-shrugging-medium-skin-tone} \\
            V & Mandeville & DES 2023 & \emoji{man-shrugging-medium-skin-tone}\\
            VI & Retraining & KBS 2024 &  \emoji{check-mark-button} \\
            VII & TextReuse & VLDB 2024 & \emoji{check-mark-button} \\

            
        \end{tabular}
    \end{table}
\end{frame}


\begin{frame}
    \frametitle{Research Projects}
    % \begin{block}{\centering Broad Research Interests}
    %     \begin{center}
    %         Scalable Pipelines for Machine Learning and Data Science
    %     \end{center}
    % \end{block}

    Multiple Projects:
    \begin{enumerate}
        \item Maintaining ML models
        \begin{itemize}
            \item Paper I: Unlearning 
            \item Paper VI: Retraining
        \end{itemize}
        \item Analyzing Historical Documents 
        \begin{itemize}
            \item Paper IV: ReceptionReader
            \item Paper V: Mandeville
            \item Paper VII: TextReuse
        \end{itemize}
        \item Scaling and Evaluating Algorithms
        \begin{itemize}
            \item Paper II: JANE
            \item Paper III: Sketching 
            \item Paper VIII    ?: Diverse Sampling 
        \end{itemize}
    \end{enumerate}

\end{frame}


\begin{frame}
    \frametitle{Maintaining ML models}

    \begin{block}{\centering Research Question}
        \centering
        How to update a trained ML model when the data changes?
    \end{block}

    \begin{enumerate}
        \item Machine Unlearning 
        \begin{itemize}
            \item Training data is deleted/removed
            \item Update model parameters to forget information
            \item \citet{mahadevan2022certifiable}
        \end{itemize}
        \item Cost-Aware Retraining
        \begin{itemize}
            \item Streams drift over time
            \item Data and Queries are present
            \item Retraining consumes resources
            \item When is it worth retraining?
            \item \citet{mahadevan2023costeffective}
        \end{itemize}
    \end{enumerate}
\end{frame}


\begin{frame}
    \frametitle{Machine Unlearning }
    \scalebox{0.5}{
    \begin{tikzpicture}
        
\usetikzlibrary{shapes.geometric}
\usetikzlibrary{calc,positioning,shapes.geometric,arrows.meta}

% positioning style from https://tex.stackexchange.com/questions/102250/how-to-position-one-node-relative-to-another-node-at-a-certain-angle-in-tikz-tak
\tikzset{
    position/.style args={#1:#2 from #3}{
            at=(#3.#1), anchor=#1+180, shift=(#1:#2)
        }
}

\tikzset{
    lines/.style={-{Latex[scale=1.5]},very thick},
    boxes/.style={very thick,
            draw=black,
            rounded corners,
            inner sep=0.5cm,
            align=center,
            rectangle,
        },
    decision/.style={very thick,
            draw,
            diamond,
            aspect=2,
            align=center,
            inner sep=0.1cm,
        },
        label/.style={very thick,
            circle,
            draw=black,
            inner sep = 0.1cm
        },
    learning/.style={draw=blue!80!black},
    unlearning/.style={draw=red!80!black}
}
\node[align=center] (origin) at (0,0) {};
\node[align=center,
    below=2cm of origin, ] (traindata) {Training\\ data \\ \D := \Dtrain} ;
\node [boxes,
    right=0.5cm of origin,
    learning, ] (training) {Training\\Stage};
\node [
    label,
    above=0.1cm of training,
    learning,
] (step1) {1};
\node [
    boxes,
    right=2cm of training,
    learning,
] (inference) {Inference\\Stage};
\node [
    label,
    above=0.1cm of inference,
    learning
] (step2) {2};
% decision diamond
\node [
    decision,
    right=1.5cm of inference,
    unlearning,
] (deletion) {Deletion?};
\node [
    right=1.5cm of deletion,
    boxes,
    unlearning,
] (unlearning) {Unlearning\\Stage};
\node [
    above=0.1cm of unlearning,
    label,
    unlearning
] (step3) {3};
% decision diamond
\node [
    decision,
    below=2cm of unlearning,
    unlearning
] (employ) {Employ\\Model?};

% lines
\draw[lines,learning] (traindata) ++(-0.5cm,1cm) |- (training); %node[midway,above=0.05cm] {\D:=\Dtrain};
\draw[lines,learning] (training) -- (inference) node[midway,above] {\wEmployed:=\worig};
\draw[lines,unlearning] (inference) -- (deletion);
\draw[lines,unlearning]  (deletion) -- node[pos=0.5,left=0.05cm] {No} ++(0,-1.5cm) -| ($(inference.south west)!.8!(inference.south east)$) ;
\draw[lines,unlearning] (deletion) -- (unlearning) node[pos=0.5,above=0.05] {Yes} node[pos=0.5,below=0.05cm] {\Dm};
\draw[lines,unlearning] (unlearning) -- (employ) node[midway,left=0.05cm] {\wunlearned};
\draw[draw=none] (unlearning) -- (employ) node[midway,right=0.05cm] (rightnode)  {\D:=\Dprime};
\draw[lines,unlearning]  (employ.west) -| ($(inference.south west)!.2!(inference.south east)$) node[pos=0.01,above=0.05cm] {Yes} ++(0,0cm)  node[pos=0.7,left] {\wEmployed:=\wunlearned};
\draw[lines,unlearning]  (employ.south) -| (training.south) node[pos=0.01,below] {No} node[pos=0.25,below]{Retrain using \D};

\coordinate (middle) at (current bounding box.center);

% \draw[draw=none] (current bounding box.west) ++(-0.5cm,0cm) -- ++(-2cm,0) node[midway,above=0.5cm,blue] {Learning};

% \draw[draw=none] (current bounding box.east) ++(0.5cm,0) -- ++(2cm,0) node[midway,above=0.5cm,red] {Unlearning};


\matrix [draw,very thick,below left,nodes={align=center},font=\small,anchor=west](m) at (current bounding box.east) {
  \node {\w:}; &\node {Employed model}; \\
  \node {\worig:}; &\node {(Re)Trained model}; \\
  \node {\wunlearned:}; &\node {Updated model}; \\
  \node {\D:}; &\node {Current data}; \\
  \node {\Dm:}; &\node {Deleted Data}; \\
  \node[lines,learning,inner sep=0,minimum width=10mm] at ++(10mm,0) {}; &\node {Learning}; \\
  \node[lines,unlearning,inner sep=0,minimum width=10mm] at ++(10mm,0) {}; &\node {Unlearning}; \\
};
\node[anchor=south] at (m.north) {Legend};
      \end{tikzpicture}
    }

    \begin{block}{Unlearning}
        Task of updating a ML model after partial deletion of training data
    \end{block}
    Qualities of an approximate unlearning method:
  \begin{itemize}
    \item \textbf{Certifiability}: How similar are \wunlearned and \worig?
    \item \textbf{Effectiveness}: How well does \wunlearned perform?
    \item \textbf{Efficiency}: How much time to produce \wunlearned?
  \end{itemize}

\end{frame}

\begin{frame}[fragile]
    \frametitle{Cost-Aware Retraining Algorithms}
    \begin{tcbraster}[raster columns=2, raster left skip=-0.9cm,raster right skip=-0.9cm,raster before skip=0mm]
        \begin{tcolorbox}[nobeforeafter, title=Cost Matrix,,left=0mm,top=0mm,boxsep=0.5mm]
            \begin{center}
            \begin{equation*}
                \small
                C[t^\prime,t]
            \end{equation*}
            \begin{align*}
                \small
                = \begin{cases}
                    \text{\footnotesize Staleness Cost} \quad& \text{\small if } t^\prime< t\\
                    \text{\footnotesize Retraining Cost} \quad& \text{\small if } t^\prime=t\\
                    \infty \quad&  \text{\small otherwise} 
                \end{cases}
            \end{align*}
        \end{center}
        \begin{center}
            \includegraphics[width=.78\textwidth]{figs/cost_matrix.pdf}
                    \end{center}
        \end{tcolorbox}
        \begin{tcolorbox}[nobeforeafter, title=Strategy,boxsep=0.5mm,left=0mm]
            \setlength{\leftmargini}{0.4cm}
            \begin{itemize}
            \item Strategy is a set of decisions
            \item Cost of decisions is strategy cost
            \item Aim is to minimize strategy cost
        \end{itemize}
        $$S = \braces{\keep,\keep,\retrain,\retrain}$$
        \begin{center}
            \includegraphics[width=.78\textwidth]{figs/strategy.pdf}
            \end{center}
            \end{tcolorbox}
    \end{tcbraster}
\end{frame}


\begin{frame}
    \frametitle{Retraining Cost \retraincost}
    \begin{itemize}
        \item Trade-off parameter between resources \& performance
        \item Low \retraincost
        \begin{itemize}
          \item Performance is important
          \item Frequent \retrain decisions to minimize staleness cost 
        \end{itemize}
        \item High \retraincost
        \begin{itemize}
          \item Resources are important
          \item \retrain decisions when large enough drops in performance
        \end{itemize}
      \end{itemize}
        \begin{minipage}{0.5\textwidth}
          \begin{center}
              {\footnotesize Low \retraincost}\\[1mm]
              \includegraphics[width=\textwidth]{figs/low_retrain_cost.pdf}
          \end{center}
        \end{minipage}%
        \begin{minipage}{0.5\textwidth}
          \begin{center}
              {\footnotesize High \retraincost}\\[1mm]
              \includegraphics[width=\textwidth]{figs/high_retrain_cost.pdf}
          \end{center}
        \end{minipage}%
    

\end{frame}


\begin{frame}
    \frametitle{Staleness Cost}
    \begin{itemize}
        \item {\textcolor{green!50!black}{Query-aware}} performance cost of old model \Mtprime at batch \ttime
        \item Scenario 1: Low staleness ~~~ Scenario 2: High staleness
      \end{itemize}
      \begin{minipage}{0.5\textwidth}
          \begin{center}
            %   {\footnotesize Scenario 1}\\[1mm]
              \includegraphics[width=0.7\textwidth]{figs/far_queries_illustration.pdf}\\[1mm]
              \includegraphics[width=0.7\textwidth]{figs/far_queries.pdf}
            \end{center}
          \end{minipage}%
          \begin{minipage}{0.5\textwidth}
            \begin{center}
            %   {\footnotesize Scenario 2}\\[1mm]
              \includegraphics[width=0.7\textwidth]{figs/near_queries_illustration.pdf}\\[1mm]
              \includegraphics[width=0.7\textwidth]{figs/near_queries.pdf}
          \end{center}
        \end{minipage}%
    

\end{frame}
\begin{frame}
    \frametitle{Analyzing Historical Documents}
    Data:
    \begin{itemize}
        \item 250k books and 1M newspapers from the 17th and 18th century
        \item Tons of heterogeneous metadata
        \begin{itemize}
            \item Collections of books and articles
            \item Publisher details
            \item Author information
        \end{itemize}
        \item Multi-modal data  
        \begin{itemize}
            \item Scanned page images
            \item OCR text
            \item XML style structured page layouts
        \end{itemize}
    \end{itemize}
    Task:
    \begin{itemize}
        \item General exploration
        \item Reception studies with Text Reuses   
        \item Top quotes of authors
    \end{itemize}
\end{frame}

\begin{frame}
    \frametitle{Reception Studies with Text Reuse}
    \includegraphics[width=\textwidth]{figs/Reception-Reader-Example.png}


\end{frame}

\begin{frame}[fragile]
    \frametitle{Identifying Text Reuse}


        However, OCR texts are very noisy
        \begin{block}{\centering Document 1 string}
            \footnotesize
            \begin{verbatim}
            to deny the Ma-
            tifl:ate a Worflip, or take away a
            hational Church, is as mere En-
            Ihufiafin as the Notion which sets
            uip Persecution    
            \end{verbatim}
        \end{block}    
        \begin{block}{\centering Document 2 string}
            \footnotesize
            \begin{verbatim}
            , to deny the Ala-
            inrate a ITorftip, or take awvay a National
            'uircb, is as mere Entnztfiafm as the Notion
            .bic fJets tup Persecution. W
            \end{verbatim}
        \end{block}    
        How to identify text reuses?
        \begin{itemize}
            \item Use BLAST to do fuzzy alignment
        \end{itemize}
\end{frame}

\begin{frame}[fragile]
    \frametitle{Pre-Processing Pipeline}
    \scalebox{0.5}{
    \input{figs/textreuse-pipeline.tex}
    }
    \begin{itemize}
        \item Clean up BLAST hits for downstream tasks
        \item Implemented in Apache Spark 
        \item Uses Dagster for asset management
        \item Scales up to {\bf 6.31 billion} pairs of reuses
    \end{itemize}
\end{frame}

\begin{frame}
    \frametitle{Related Publications}

    \begin{enumerate}
        \item Paper IV: \citet{Rosson-2023}
        \begin{itemize}
            \item \emph{Reception reader: Exploring text reuse in early modern british publications}
            \item Front-end user interface for browsing reuses
        \end{itemize}
        \item Paper V: \citet{des6}
        \begin{itemize}
            \item \emph{A Comparative text similarity analysis of the works of Bernard Mandeville}
            \item Study using the data and interfaces from Paper IV
        \end{itemize}
        \item Paper VI: \citet{mahadevan2024optimizing}
        \begin{itemize}
            \item \emph{Optimizing a Data Science System for Text Reuse Analysis}
            \item Studies design choices to optimize performance of the system
            \item Plans to scale interface from Paper IV based on insights
        \end{itemize}
    \end{enumerate}


\end{frame}
\begin{frame}
    \frametitle{Scalability and Robustness}
    \begin{itemize}
        \item Paper II: \citet{merchant2022JANE}
        \begin{itemize}
            \item Scaled up GNN alternative from \citet{merchant2022JANEorig} to run effectively on graphs with millions of nodes
        \end{itemize}
        \item Paper III: \citet{10.1145/3511808.3557687}
        \begin{itemize}
            \item Explored the robustness of Sketched Linear Networks from \citet{tai2018sketch} to adversarial attacks
        \end{itemize}
        \item Paper VIII?: 
        \begin{itemize}
            \item Original algorithm from \citet{wang2023fmmd}
            \item Re-implemented for edge-case requirement in Historical Project
            \item Nearly $200\times$ speed-up compared to original code
            \item Plans to develop theory and code for distributed algorithm 
        \end{itemize}
    \end{itemize}
\end{frame}

\begin{frame}
    \frametitle{Next Steps}
    \begin{enumerate}
        \item Complete a few more transferrable skill credits
        \item Start Writing Thesis
        \item Work on Paper VIII in tandem 
    \end{enumerate}
    

\end{frame}

\begin{frame}[allowframebreaks]
    \frametitle{References}
    \bibliographystyle{plainnat}
    \bibliography{references.bib}
  \end{frame}

\end{document}