
\usetikzlibrary{shapes.geometric}
\usetikzlibrary{calc,positioning,shapes.geometric,arrows.meta}

% positioning style from https://tex.stackexchange.com/questions/102250/how-to-position-one-node-relative-to-another-node-at-a-certain-angle-in-tikz-tak
\tikzset{
    position/.style args={#1:#2 from #3}{
            at=(#3.#1), anchor=#1+180, shift=(#1:#2)
        }
}

\tikzset{
    lines/.style={-{Latex[scale=1.5]},very thick},
    boxes/.style={very thick,
            draw=black,
            rounded corners,
            inner sep=0.5cm,
            align=center,
            rectangle,
        },
    decision/.style={very thick,
            draw,
            diamond,
            aspect=2,
            align=center,
            inner sep=0.1cm,
        },
        label/.style={very thick,
            circle,
            draw=black,
            inner sep = 0.1cm
        },
    learning/.style={draw=blue!80!black},
    unlearning/.style={draw=red!80!black}
}
\node[align=center] (origin) at (0,0) {};
\node[align=center,
    below=2cm of origin, ] (traindata) {Training\\ data \\ \D := \Dtrain} ;
\node [boxes,
    right=0.5cm of origin,
    learning, ] (training) {Training\\Stage};
\node [
    label,
    above=0.1cm of training,
    learning,
] (step1) {1};
\node [
    boxes,
    right=2cm of training,
    learning,
] (inference) {Inference\\Stage};
\node [
    label,
    above=0.1cm of inference,
    learning
] (step2) {2};
% decision diamond
\node [
    decision,
    right=1.5cm of inference,
    unlearning,
] (deletion) {Deletion?};
\node [
    right=1.5cm of deletion,
    boxes,
    unlearning,
] (unlearning) {Unlearning\\Stage};
\node [
    above=0.1cm of unlearning,
    label,
    unlearning
] (step3) {3};
% decision diamond
\node [
    decision,
    below=2cm of unlearning,
    unlearning
] (employ) {Employ\\Model?};

% lines
\draw[lines,learning] (traindata) ++(-0.5cm,1cm) |- (training); %node[midway,above=0.05cm] {\D:=\Dtrain};
\draw[lines,learning] (training) -- (inference) node[midway,above] {\wEmployed:=\worig};
\draw[lines,unlearning] (inference) -- (deletion);
\draw[lines,unlearning]  (deletion) -- node[pos=0.5,left=0.05cm] {No} ++(0,-1.5cm) -| ($(inference.south west)!.8!(inference.south east)$) ;
\draw[lines,unlearning] (deletion) -- (unlearning) node[pos=0.5,above=0.05] {Yes} node[pos=0.5,below=0.05cm] {\Dm};
\draw[lines,unlearning] (unlearning) -- (employ) node[midway,left=0.05cm] {\wunlearned};
\draw[draw=none] (unlearning) -- (employ) node[midway,right=0.05cm] (rightnode)  {\D:=\Dprime};
\draw[lines,unlearning]  (employ.west) -| ($(inference.south west)!.2!(inference.south east)$) node[pos=0.01,above=0.05cm] {Yes} ++(0,0cm)  node[pos=0.7,left] {\wEmployed:=\wunlearned};
\draw[lines,unlearning]  (employ.south) -| (training.south) node[pos=0.01,below] {No} node[pos=0.25,below]{Retrain using \D};

\coordinate (middle) at (current bounding box.center);

% \draw[draw=none] (current bounding box.west) ++(-0.5cm,0cm) -- ++(-2cm,0) node[midway,above=0.5cm,blue] {Learning};

% \draw[draw=none] (current bounding box.east) ++(0.5cm,0) -- ++(2cm,0) node[midway,above=0.5cm,red] {Unlearning};


\matrix [draw,very thick,below left,nodes={align=center},font=\small,anchor=west](m) at (current bounding box.east) {
  \node {\w:}; &\node {Employed model}; \\
  \node {\worig:}; &\node {(Re)Trained model}; \\
  \node {\wunlearned:}; &\node {Updated model}; \\
  \node {\D:}; &\node {Current data}; \\
  \node {\Dm:}; &\node {Deleted Data}; \\
  \node[lines,learning,inner sep=0,minimum width=10mm] at ++(10mm,0) {}; &\node {Learning}; \\
  \node[lines,unlearning,inner sep=0,minimum width=10mm] at ++(10mm,0) {}; &\node {Unlearning}; \\
};
\node[anchor=south] at (m.north) {Legend};