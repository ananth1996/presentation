
\DeclareMathOperator*{\argmin}{\arg\!\min}
\DeclareMathOperator*{\argmax}{\arg\!\max}
\usepackage{bm}
\usepackage{tikz}
\usepackage{xspace}
\usetikzlibrary{positioning,arrows}
\usepackage{xcolor}

\newcommand{\param}{\boldsymbol{\theta}}
\newcommand{\ex}{p}
\newcommand{\exapp}{\tilde{p}}
\newcommand{\posapp}{q}
\newcommand{\un}{q_u}
\newcommand{\unapp}{\tilde{q}_u}
\newcommand{\rklapp}{\tilde{q}_v}
\newcommand{\data}{\mathcal{D}}
\newcommand{\erase}{\mathcal{D}_e}
\newcommand{\remain}{\mathcal{D}_r}

\definecolor{uablue}{RGB}{0,61,100}
\colorlet{uablue100}{uablue}
\colorlet{uablue75} {uablue!75!white}
\colorlet{uablue50} {uablue!50!white}
\colorlet{uablue25} {uablue!25!white}
\colorlet{uablue10} {uablue!10!white}
\colorlet{uablue5}  {uablue!5!white}

\definecolor{uared}{RGB}{126,0,47}
\colorlet{uared100}{uared}
\colorlet{uared75} {uared!75!white}
\colorlet{uared50} {uared!50!white}
\colorlet{uared25} {uared!25!white}
\colorlet{uared10} {uared!10!white}
\colorlet{uared5}  {uared!5!white}

\tikzset{onslide/.code args={<#1>#2}{%
  \only<#1>{\pgfkeysalso{#2}} % \pgfkeysalso doesn't change the path
}}
\tikzset{alt/.code args={<#1>#2#3}{%
  \alt<#1>{\pgfkeysalso{#2}}{\pgfkeysalso{#3}} % \pgfkeysalso doesn't change the path
}}
\tikzset{temporal/.code args={<#1>#2#3#4}{%
  \temporal<#1>{\pgfkeysalso{#2}}{\pgfkeysalso{#3}}{\pgfkeysalso{#4}} % \pgfkeysalso doesn't change the path
}}
% positioning style from https://tex.stackexchange.com/questions/102250/how-to-position-one-node-relative-to-another-node-at-a-certain-angle-in-tikz-tak
\tikzset{
    position/.style args={#1:#2 from #3}{
        at=(#3.#1), anchor=#1+180, shift=(#1:#2)
    }
}

\newcommand{\mbf}[1]{\mathbf{#1}}
\newcommand{\mcl}[1]{\mathcal{#1}}
\newcommand{\mbb}[1]{\mathbb{#1}}

\newcommand{\da}{\mcl{D}}
\newcommand{\dc}{\mcl{D}_r}
\newcommand{\dr}{\mcl{D}_e}

\newcommand{\eubo}{\tilde{q}_u}
\newcommand{\elbo}{\tilde{q}_v}

