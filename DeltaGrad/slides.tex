\documentclass{beamer}
\usetheme{Madrid}

\newcommand{\smartart}{   
    \begin{center}
    \scalebox{0.68}{   
        \smartdiagram[constellation diagram:counterclockwise]{
        Updating ML Models,Differential\\Privacy, Optimization, Information\\Theory, Novel\\Pipelines
        }   
    }
    \end{center}
}
  
\newcommand\myNset[1][6]{\ifnum#1=1 \smartdiagramset{set color list={red!60,blue!60,blue!60,blue!60,blue!60},
  uniform connection color=true,
  distance planet-satellite=4.1cm,
  satellite text width=2.5cm,
  planet text width=2.7cm, 
  /tikz/connection planet satellite/.append style={->}
  }
  \else\ifnum#1=2 \smartdiagramset{set color list={blue!60,red!60,blue!60,blue!60,blue!60},
  uniform connection color=true,
  distance planet-satellite=4.1cm,
  satellite text width=2.5cm,
  planet text width=2.75cm, 
  /tikz/connection planet satellite/.append style={->}
  }
  \else\ifnum#1=3 \smartdiagramset{set color list={blue!60,blue!60,red!60,blue!60,blue!60},
  uniform connection color=true,
  distance planet-satellite=4.1cm,
  satellite text width=2.5cm,
  planet text width=2.75cm, 
  /tikz/connection planet satellite/.append style={->}
  }
  \else\ifnum#1=4 \smartdiagramset{set color list={blue!60,blue!60,blue!60,red!60,blue!60},
  uniform connection color=true,
  distance planet-satellite=4.1cm,
  satellite text width=2.5cm,
  planet text width=2.75cm, 
  /tikz/connection planet satellite/.append style={->}
  }
  \else\ifnum#1=5 \smartdiagramset{set color list={blue!60,blue!60,blue!60,blue!60,red!60},
  uniform connection color=true,
  distance planet-satellite=4.1cm,
  satellite text width=2.5cm,
  planet text width=2.75cm, 
  /tikz/connection planet satellite/.append style={->}
  }
  \else\ifnum#1=6 \smartdiagramset{set color list={blue!60,blue!60,blue!60,blue!60,blue!60},
  uniform connection color=true,
  distance planet-satellite=4.1cm,
  satellite text width=2.5cm,
  planet text width=2.75cm, 
  /tikz/connection planet satellite/.append style={->}
  }
  
  \fi\fi\fi\fi\fi\fi}


\newcommand{\dataset}{\mathcal{D}}
\newcommand{\datasetprime}{\mathcal{D}^{\prime}}
\newcommand{\removed}{\mathcal{D_{R}}}
\newcommand{\alg}{A}
\newcommand{\mech}{M}
\newcommand{\hypothesis}{\mathcal{H}}
\newcommand{\subhypothesis}{\mathcal{T}}
\newcommand{\wopt}{\textbf{w}^{*}}
\newcommand{\w}{\textbf{w}}
\newcommand{\wu}{\textbf{w}^{U}}
\newcommand{\wi}{\textbf{w}^{I}}
\newcommand{\h}{\textbf{H}}
\newcommand{\x}{\textbf{x}}
\newcommand{\wminus}{\textbf{w}^{-}}
\newcommand{\risk}{L}
\newcommand{\loss}{\ell}

\DeclareMathOperator*{\argmin}{\arg\!\min}
\DeclareMathOperator*{\argmax}{\arg\!\max}


\usepackage{multicol, latexsym, amsmath, amssymb}
\usepackage{appendixnumberbeamer} 

\setbeamertemplate{navigation symbols}{}


\AtBeginSection[]{
  \begin{frame}
  \vfill
  \centering
  \begin{beamercolorbox}[sep=8pt,center,shadow=true,rounded=true]{title}
    \usebeamerfont{title}\insertsectionhead\par%
  \end{beamercolorbox}
  \vfill
  \end{frame}
}

\AtBeginSubsection[]{
  \begin{frame}
  \vfill
  \centering
  \begin{beamercolorbox}[sep=8pt,center,shadow=true,rounded=true]{title}
    \usebeamerfont{title}\insertsubsectionhead\par%
  \end{beamercolorbox}
  \vfill
  \end{frame}
}



\usepackage{subcaption}

\title[DeltaGrad]{Weekly Presentation\\DeltaGrad: Rapid retraining of machine learning models}

\author[Wu et al.~]{Yinjun Wu \and Edgar Dobriban \and Susan B Davidson}
\date{\today}

\begin{document}
\begin{frame}
    \titlepage
\end{frame}

\begin{frame}
    \frametitle{Overview}
    \tableofcontents
\end{frame}


\section{Motivation}
\begin{frame}
    \frametitle{Retaining Problem}
    Regular Pipeline:
    \begin{enumerate}
      \item Train a ML model from data using a learning algorithm
      \item Small change in training data occurs (deletions or additions)
      \item Retrain ML model from scratch
    \end{enumerate}

    \begin{itemize}
      \item<2-> Computationally expensive process
      \item<3-> Throws away useful computations from initial training
    \end{itemize}
    \begin{block}<4->{Research Question}
      Can we retrain models in an efficient manner?
    \end{block}
\end{frame}

\begin{frame}
  \frametitle{Potential Applications}
  \begin{itemize}
    \item<1-> \textbf{GDPR}: Deletion of private information from public datasets
    \item<2-> \textbf{Continuous Model Updating}: Handle additions, deletions and changes of training samples
    \item<3-> \textbf{Data Valuation}: \textit{Leave One Out} tests to find important training samples 
    \item<4-> \textbf{Bias Reduction}: Speeds up jackknife resampling that requires retrained model parameters 
  \end{itemize}
  

\end{frame}
\section{Related Work}

\begin{frame}
  \frametitle{Prior Work}
  \begin{itemize}
    \item Prior work for specialized problems and ML models, usually for deletion
    \begin{itemize}
      \item Provenane Based deletions for linear and logistic regression \cite{wuPrIUProvenanceBasedApproach2020}
      \item Newton step and noise for \textit{certified data removal} \cite{guoCertifiedDataRemoval2020}
      \item K-means clustering \cite{ginartMakingAIForget2019}
    \end{itemize}
  \end{itemize}
\end{frame}

\section{DeltaGrad}
\begin{frame}
  \frametitle{Gradient Descent}
  \begin{itemize}
    \item Objective function  
    \[
        F(\w) = \frac{1}{n}\sum_{i=1}^{n}F_{i}(\w)  
    \]
    \item Stochastic Gradient Descent update rule, $\mathcal{B}_{t}$ is randomly sampled mini-batch of size $B$
    \[
      \w_{t+1} \leftarrow \w_{t}-\frac{\eta_{t}}{B} \sum_{i \in \mathcal{B}_{t}} \nabla F_{i}\left(\w_{t}\right) 
    \]
    \item Full-batch gradient descent (GD) is on entire data 
    \[
      \w_{t+1} \leftarrow \w_{t}-\frac{\eta_{t}}{n} \sum_{i=1}^{n} \nabla F_{i}\left(\w_{t}\right) 
    \]
  \end{itemize} 
\end{frame}

\begin{frame}
  \frametitle{Removal of data}
  \begin{itemize}
    \item After training,  $R = \{i_1,i_2,\dots,i_r\}$ is removed, where $r \ll n$
    \item Naive retraining is applying GD over remaining samples, $\w^{U}$ is resulting parameters
    \begin{align}\label{eq: update_rule_naive} 
      & \uw_{t+1} \leftarrow \uw_{t} - \frac{\eta_t}{n-r}\sum_{\substack{i \not\in R}} \nabla F_i\left(\uw_{t}\right) 
    \end{align}
    \item The explicit gradient computation $\sum_{\substack{i \not\in R}} \nabla F_i\left(\uw_{t}\right)$ is expensive
    \item Instead rewrite \eqref{eq: update_rule_naive} as follows 
    \begin{align}\label{eq: approx_w_t}
      \begin{split}
      \uw_{t+1} 
               = \uw_{t} - \frac{\eta_t}{n-r}\left[\sum_{i=1}^{n} \nabla F_{i}\left(\uw_{t}\right) - \sum_{\substack{i \in R}} \nabla F_i\left(\uw_{t}\right)\right].
      \end{split}
      \end{align}
      \item $\sum_{\substack{i \in R}} \nabla F_i\left(\uw_{t}\right)$ is cheaper to compute
  \end{itemize}
\end{frame}

\begin{frame}
  \frametitle{Etymology}
  \begin{itemize}
    \item After a small change to the data we need to redo the SGD computations
    \item We can achieve this by understanding the small \textit{delta} of the Gradient Descent 
    \[ 
      \nabla F(\w) = \sum_{i=1}^{n}\nabla F_{i}(\w_t) \quad \& \quad \nabla F(\uw) =  \sum_{i=1}^{n} \nabla F_{i}\left(\uw_{t}\right)
    \]
    \item Hence, the approach is called \textit{DeltaGrad}
  \end{itemize}
\end{frame}

\begin{frame}
  \frametitle{Aprroximating $\nabla F(\uw)$}
  \begin{itemize}
    \item $\w_{0}$, $\dots,\w_{t}$ and $\nabla F\left(\w_{0}\right)$, $\dots, \nabla F\left(\w_{t}\right)$ are cached from training on initial dataset
    \item By Cauchy mean-value theorem\footnote<2->{Seems to be a consequence of Fundamental theory of Calculus and mean-value theorem}
    \[
      \nabla F(\uw_{t}) - \nabla F(\w_{t}) = \bH_{t} \cdot (\uw_{t} - \w_{t}) 
    \]
    Where $\bH_{t} = \int_{0}^{1}\bH(\w_{t}+x(\uw_{t}-\w_{t}))dx$ is the integrated hessian
    \item This requires a hessian $\bH_{t}$ at each step, which is expensive to maintain and evaluate
    \item Leverage classical L-BFGS algorithm to approximate $\bH_{t}$
  \end{itemize}
\end{frame}
\section{Theoretical Results}

\section{Experimental Results}

\bibliography{UpdateMl}
\bibliographystyle{alpha}

\appendix
\section{Large Deletions}
\begin{frame}
  \frametitle{Large Deletions}


\end{frame}



\end{document}  