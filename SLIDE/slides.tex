\documentclass{beamer}
\usetheme{Madrid}
\AtBeginSection[]{
  \begin{frame}
  \vfill
  \centering
  \begin{beamercolorbox}[sep=8pt,center,shadow=true,rounded=true]{title}
    \usebeamerfont{title}\insertsectionhead\par%
  \end{beamercolorbox}
  \vfill
  \end{frame}
}

\AtBeginSubsection[]{
  \begin{frame}
  \vfill
  \centering
  \begin{beamercolorbox}[sep=8pt,center,shadow=true,rounded=true]{title}
    \usebeamerfont{title}\insertsubsectionhead\par%
  \end{beamercolorbox}
  \vfill
  \end{frame}
}



\usepackage{subcaption}
\usepackage{tikz}
\usetikzlibrary{positioning} 
\usetikzlibrary{matrix}
\usetikzlibrary{calc}

\newcommand{\smartart}{   
    \begin{center}
    \scalebox{0.68}{   
        \smartdiagram[constellation diagram:counterclockwise]{
        Updating ML Models,Differential\\Privacy, Optimization, Information\\Theory, Novel\\Pipelines
        }   
    }
    \end{center}
}
  
\newcommand\myNset[1][6]{\ifnum#1=1 \smartdiagramset{set color list={red!60,blue!60,blue!60,blue!60,blue!60},
  uniform connection color=true,
  distance planet-satellite=4.1cm,
  satellite text width=2.5cm,
  planet text width=2.7cm, 
  /tikz/connection planet satellite/.append style={->}
  }
  \else\ifnum#1=2 \smartdiagramset{set color list={blue!60,red!60,blue!60,blue!60,blue!60},
  uniform connection color=true,
  distance planet-satellite=4.1cm,
  satellite text width=2.5cm,
  planet text width=2.75cm, 
  /tikz/connection planet satellite/.append style={->}
  }
  \else\ifnum#1=3 \smartdiagramset{set color list={blue!60,blue!60,red!60,blue!60,blue!60},
  uniform connection color=true,
  distance planet-satellite=4.1cm,
  satellite text width=2.5cm,
  planet text width=2.75cm, 
  /tikz/connection planet satellite/.append style={->}
  }
  \else\ifnum#1=4 \smartdiagramset{set color list={blue!60,blue!60,blue!60,red!60,blue!60},
  uniform connection color=true,
  distance planet-satellite=4.1cm,
  satellite text width=2.5cm,
  planet text width=2.75cm, 
  /tikz/connection planet satellite/.append style={->}
  }
  \else\ifnum#1=5 \smartdiagramset{set color list={blue!60,blue!60,blue!60,blue!60,red!60},
  uniform connection color=true,
  distance planet-satellite=4.1cm,
  satellite text width=2.5cm,
  planet text width=2.75cm, 
  /tikz/connection planet satellite/.append style={->}
  }
  \else\ifnum#1=6 \smartdiagramset{set color list={blue!60,blue!60,blue!60,blue!60,blue!60},
  uniform connection color=true,
  distance planet-satellite=4.1cm,
  satellite text width=2.5cm,
  planet text width=2.75cm, 
  /tikz/connection planet satellite/.append style={->}
  }
  
  \fi\fi\fi\fi\fi\fi}


\newcommand{\dataset}{\mathcal{D}}
\newcommand{\datasetprime}{\mathcal{D}^{\prime}}
\newcommand{\removed}{\mathcal{D_{R}}}
\newcommand{\alg}{A}
\newcommand{\mech}{M}
\newcommand{\hypothesis}{\mathcal{H}}
\newcommand{\subhypothesis}{\mathcal{T}}
\newcommand{\wopt}{\textbf{w}^{*}}
\newcommand{\w}{\textbf{w}}
\newcommand{\wu}{\textbf{w}^{U}}
\newcommand{\wi}{\textbf{w}^{I}}
\newcommand{\h}{\textbf{H}}
\newcommand{\x}{\textbf{x}}
\newcommand{\wminus}{\textbf{w}^{-}}
\newcommand{\risk}{L}
\newcommand{\loss}{\ell}

\DeclareMathOperator*{\argmin}{\arg\!\min}
\DeclareMathOperator*{\argmax}{\arg\!\max}


\usetikzlibrary{shadows,positioning}
\colorlet{colD}{red!40}
\colorlet{colIP}{cyan!40}
\colorlet{colV}{blue!40}
\colorlet{colBorder}{gray!70}


\title[SLIDE]{MLDB Presentation\\SLIDE: Sub-LInear Deep Learning Engine}

\author[Beidi Chen et al.]{Beidi Chen \and Tharun Medini \and James Farwell \and Sameh Gobriel \and Charlie Tai \and Anshumali Shrivastava}
\date{\today}

\begin{document}
\begin{frame}
    \titlepage
\end{frame}

\begin{frame}
    \frametitle{Overview}
    \tableofcontents
\end{frame}


\section{Motivation}
\begin{frame}
    \frametitle{Era of Deep Learning}
    \begin{figure}[ht!]
        \centering
        \begin{subfigure}{0.4\textwidth}
            \centering
            \includegraphics[width=1.2\textwidth]{images/deep learning.png}
            \caption{Model Performance wrt Dataset size}
            \label{fig:deep-learning}
        \end{subfigure}
        \begin{subfigure}{0.4\textwidth}
            \centering
            \includegraphics[width=0.7\textwidth]{images/xkcd.png}
            \caption{Fun xkcd comic}
            \label{fig:deep-learning}
        \end{subfigure}

    \end{figure}
    
\end{frame}
\begin{frame}
    \frametitle{Trends}
    \begin{itemize}
        \item Large datasets $\rightarrow$ More Data
        \item Big models (Eg, 17B parameter NLP models)
        \item Improvements in optimizations and gradient descent
        \item \structure{Matrix multiplication} is a computational bottleneck
        \item Many approaches exists such as \structure{GPUs}
    \end{itemize}

\end{frame}

\subsection{Existing Approaches}
\begin{frame}
    \frametitle{Low Rank structure}

    \begin{minipage}{0.6\textwidth}
        \begin{itemize}
            \item $W \in \mathbb{R}^{m\times n}$ is weight matrix
            \item $W$ has a low-rank structure $W=UV$
            \item $U\in \mathbb{R}^{m\times r}$ and $V\in \mathbb{R}^{r\times n}$, where $r \ll \min(m,n)$
            \item Equivalent representation with $I$ activation function is better
            \item $\mathcal{O}(mn)$ becomes $\mathcal{O}(mr+rn)$
            \item \structure{Better storage} of parameters as well
            \item But still needs dense gradient update, cannot parallelise asynchronously
        \end{itemize}
    \end{minipage}\hfill
    \begin{minipage}{0.4\textwidth}
        \includegraphics[width=\textwidth]{images/Low_Rank.png}
    \end{minipage}

\end{frame}

\begin{frame}
    \frametitle{Dropout and Sparsity}
    \begin{minipage}{0.5\textwidth}
        \begin{itemize}
            \item Well known regularization method for Neural Networks
            \item With probability $p$ neurons in each layer is \structure{turned off}
            \item Used during training to ensure model generalizes 
            \item Sparsity above 50\% tends to begin hurting performance
        \end{itemize}
    \end{minipage}\hfill
    \begin{minipage}{0.5\textwidth}
        \includegraphics[width=\textwidth]{images/dropout.png}
    \end{minipage}
    
\end{frame}

\begin{frame}
    \frametitle{Adaptive Dropout}
    \begin{itemize}
        \item \cite{Lei_adaptive_dropout}
    \end{itemize}
    

\end{frame}
\subsection{Problem Setting}

\section{Contributions}
\begin{frame}
    \frametitle{Main Contributions}

    \begin{itemize}
        \item C++ OpenMP implementation for "standard" CPU
        \item Sparsity inspired, LSH based backpropagation algorithm
        \item Rigorous evaluation with TF-GPU and CPU
        \item Further optimizations using Hugepages and SIMD instructions
    \end{itemize}
    

\end{frame}

\section{Locality Sensitive Hashing}
\subsection{Sampling Approach to LSH}
\subsection{Additional Optimizations}

\section{Implementation}



\section{Results}

\begin{frame}
    \frametitle{}

    \centering \Large
    \emph{Questions or Comments}

\end{frame}

\bibliography{sources}
\bibliographystyle{apalike}
\end{document}